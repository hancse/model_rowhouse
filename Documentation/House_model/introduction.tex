% \chapter{Introduction}

\section{Introduction}

Building energy simulation is a vast field of research that started on the late 50’s and that is still highly active nowadays. Building energy simulations are mainly used to help taking design decisions, to analyze current designs and to forecast future building energy use. Building energy modelling methods can mainly be divided into three categories:
\begin{itemize}
  
    \item White box model (physics-base)
    \item Black box model (data-driven)
    \item Grey box (hybrid)

\end{itemize}


White box model is based on the equations related to the fundamental laws of energy and mass balance and heat transfer. White box models can be differentiated in two types, distributed parameter models and lumped parameter models. Lumped parameter models simplify the description of distributed physical systems into discrete entities that approximate the behavior of a distributed system. The advantage of using lumped models is the decrease on simulation time (Ramallo-González et al.). White box model is of special interest for the design phase as they are used to predict and analyses the performance of the building envelope and building systems.
Black box models are based on the statistical relation between input and output system values. The statistical relation between input and output is based on actual data. The relation between the parameters can differ based on the amount of data and the method used to analyze the relation. Currently, there is a large and active field of research about statistical models that are used on black box models (Coacley et al.). Black box models are of special interest when there is a large amount of actual input and output data available.  
Grey box model is a hybrid model form that aim to combine the advantages of both systems. In order to use them it is necessary to implement some equations and it is also required to have actual data of inputs and outputs.

\newpage
